% Coding Style, or The Way I like It =)
% $Id: style.tex,v 1.1.1.1 2002/08/20 10:03:08 berkus Exp $

\documentclass[onecolumn]{article}

\begin{document}
\date{}
\title{\Large\bf Odin OS Source Code Style}
\author{Stanislav Karchebny, \\
        email: \texttt{berk@madfire.net}}
\maketitle



\section{Foreword}



\section{Common considerations}


\subsection{Tabulations (TABs)}

\par I use tabs of 3 spaces (a good balance between unreadable 2 spaces and too large 8 spaces =).
Some people use 2, some 4 and some 8 spaces per tab. There's an issue though - formatted text may
be destroyed by change of tab size. So here's what I require:

\par Tabs at the beginning of the line (before any non-whitespace character) are allowed and encouraged
(iow, you should use tabs instead of spaces here!!!), this allows people to use their preferred tab
size and see well-formatted indentation.
BUT! In the rest of the line there \emph{MUST BE NO TABS} ever. If you line up several variables, or
put a comment aligned on a specific column (see below about comments), use \emph{SPACES ONLY}.
See in this example:

\small
\begin{verbatim}
<TAB>static int<TAB>var<TAB>= 0;<TAB><TAB>//comment
<TAB>static int<TAB>variab<TAB>= 012;<TAB><TAB>//comment
\end{verbatim}
\normalsize

\par With tab set to 3 it will look like this:

\small
\begin{verbatim}
   static int  var   = 0;     //comment
   static int  variab   = 012;      //comment
\end{verbatim}
\normalsize

\par With tab set to 8 it will look like this:

\small
\begin{verbatim}
        static int      var     = 0;            //comment
        static int      variab  = 012;          //comment
\end{verbatim}
\normalsize

\par I hope you clearly see the difference! But if we use TABS only in the beginning of line,
both version will look aligned the same, only the starting column will be different.


\subsection{Line length}

\par Maximum line length is not "strictly" limited since modern terminals allow quite a wide screens.
However, for the sake of older development platforms right margin is "softly" set to 80 chars - try to
avoid longer lines.

\par "Hard" right margin is defined to be 128 chars - no line can be longer than this.



\section{C/C++ language code style}


\subsection{Block constructs}

\par Some people put opening brace on the same line as operator, some on the next.
I prefer it on the next for only one reason - I can clearly see the block boundaries.
  
\small
\begin{verbatim}
void functionname(int param1, int param2)
{
}
  
struct name
{
   int    field1;
   void  *field2;
};
\end{verbatim}
\normalsize


\subsection{Block comments}

\par This is generic function/method comment:

\small
\begin{verbatim}
/*----------------------------.
|  SelectSockets              |
|-----------------------------'---------------------------------------------.
|  int SelectSockets(int *mask,                                             |
|                    struct timeval *timeout)                               |
|                                                                           |
|  Description: calls select () to poll a set of streams or sockets for     |
|  input data.  Mask is a bitmap of streams (ie. socket handles) to check.  |
|  Timeout is the maximum poll delay, in seconds and microseconds.  Unix    |
|  usually sets a lower limit of 1/1000 second.                             |
|                                                                           |
|  Returns: 0 to MASK_MAX - 1 indicates the socket which has input ready;   |
|  SELECT_TIMEOUT indicates that no input arrived before the timeout; -1    |
|  indicates a fatal error - in that case, SelectSockets displays a message |
|  on stderr.                                                               |
|                                                                           |
|  Comments: SelectSockets does not poll for write or exception.            |
`--------------------------------------------------------------------------*/
int SelectSockets(int *mask, struct timeval *timeout)
{
}
\end{verbatim}
\normalsize


\subsection{Class description}

\small
\begin{verbatim}
class ClassName : public BaseClass
{
public: /* methods */
   ClassName() : BaseClass {}
   ClassName(char *param);
   ~ClassName() {}

   void inline_method1();

public:
   /* fields (hey! DON'T do this!) */

protected:
   /* methods */

protected:
   /* fields */

private:
   /* methods */

private:
   /* fields */
   char *data;
};

ClassName::ClassName(char *param) : BaseClass()
                                  , data(param)
{
}

void ClassName::inline_method1()
{
}
\end{verbatim}
\normalsize



\section{Interface Definition Language (IDL) style}


\subsection{Interface Definition}

\small
\begin{verbatim}
module ModuleName
{
   interface InterfaceName : base_interface_list
   {
      rettype method_name( // short description for method_name
         type param1,      // short description for param1
         type param2       // short description for param2
      );
      ....
      rettype method_2();  // short description for method_2
   };
};
\end{verbatim}
\normalsize



\section{Assembly language code style}

\par Assembly language considerations are way simpler than C/C++ ones.


\subsection{Block comments}

\par This is generic function/method comment:

\small
\begin{verbatim}
;/*----------------------------.
;|  utf8_to_ucs2               |
;|-----------------------------'---------------------------------------------.
;|  Description: Converts UTF8-encoded string to UCS2-encoded equivalent.    |
;|                                                                           |
;|  Input: ES:EDX = pointer to UTF8-encoded string                           |
;|  Output: ES:EDX = pointer to UCS2-encoded string                          |
;|          ZF set if conversion failed.                                     |
;|  Other registers unmodified                                               |
;`--------------------------------------------------------------------------*/
utf8_to_ucs2:
\end{verbatim}
\normalsize

\end{document}
