% Jotun File Format Specification, version 1.0.2.
% $Id: Jotun.tex,v 1.1.1.1 2002/08/20 10:03:09 berkus Exp $

% TODO: fix <johndoe@> style (converts <> into spanish stuff!!)
%       fix spelling Jotun correct ('o or "o instead of o!!)

\documentclass[onecolumn]{article}

\usepackage{graphicx}

\pagestyle{empty}
\setlength{\textheight}{8.75in}
\setlength{\columnsep}{2.0pc}
\setlength{\textwidth}{6.8in}
\setlength{\topmargin}{0.25in}
\setlength{\headheight}{0.0in}
\setlength{\headsep}{0.0in}
\setlength{\oddsidemargin}{-.19in}
\setlength{\parindent}{1pc} % conform to IEEE format

\begin{document}
\title{\Large\bf Odin OS \\
                 Jotun file format specification}
\author{Stanislav Karchebny, \\
        email: \texttt{berk@madfire.net}}
\maketitle



\section{Introduction}

\par Jotun file format is intended to be as small as possible, while maintaining
high scalability and portability. It is bound to features used in Odin OS -
components, method tables etc.



\section{History bits}



\section{Format}

\par Jotun file consists of two or more headers and some data.
All fields in headers are little-endian.
First header defines target architecture this file is intended to run on.
It also defines a compatibility version information.


\subsection{File Header}
\label{sec:filehdr}

\begin{center}
\par File Header - mandatory
\begin{tabular}{|l|l|l|l|}\hline
\emph{off} & \emph{size} & \emph{name} & \emph{meaning/value} \\
\hline\hline
0      & dword    & magic      & `E',`T',`i',`N' (an english version for "Jotun") \\
4*     & word     & hdrsize    & header size in bytes including magic and extra fields (min = 16) \\
6      & word     & version    & Odin version this file is produced/compatible with \\
8      & byte     & reserved   & padding, zero \\
9      & byte     & machine    & machine architecture \\
10**   & byte     & capacity   & file capacity (see below) \\
11     & byte     & encoding   & file endianness (see below) \\
12     & dword    & flags      & flags (see below) \\
\hline
\end{tabular}
\end{center}

\par File capacity
\begin{description}
\item[ 0 ] - undef
\item[ 1 ] - 32 bits
\item[ 2 ] - 64 bits
\end{description}

\par File endianness
\begin{description}
\item[ 0 ] - undef
\item[ 1 ] - little-endian (Intel)
\item[ 2 ] - big-endian (Motorola)
\end{description}

\par Flags
\begin{description}
\item[ bit 0: ] text info present
\item[ bit 1: ] overlay data sections present
\item[ bit 2: ] data section in Sections Header is unique (that is, component is singleton)
\item[ other bits ] undefined
\end{description}

\par NOTES
\begin{description}
\item[ * ] two fields at offset 4 are read as single 4 bytes little-endian
value with version in high word and header size in low word. Odin version
is 4 bytes little-endian value with major.minor in high word and build in
low word. So, to compare versions you just have to compare two high words.
Header size usually remains constant within single version, so seeing unusual
values means corrupted/wrong file.
\item[ ** ] current document describes 32 bits Jotun files. 64 bits files are likely to
have different sizes for many fields.
\end{description}

\par Following file header goes sections header: a comp-compatible list of
component sections and their offsets/sizes. (see \ref{sec:secthdr})

\par If File Header flags bit 0 is set, Sections Header is followed by Text Info,
which has format of free-form key=value dictionary style. (see \ref{sec:textinfo})

\par Then, if bit 1 in file header flags is set, follows overlay data sections header. (see \ref{sec:odshdr})

\par Then sections in no particular order follows. The usual order however is
text, data then method table.

\par If bit 2 in file header flags is set, the data section should (but not have to) be
instantiated in-place since this is "singleton" component - a component that can only
have one single instance.


\subsection{Sections Header}
\label{sec:secthdr}

\begin{center}
\par Sections Header - mandatory
\begin{tabular}{|l|l|l|l|}\hline
\emph{off} & \emph{size} & \emph{name} & \emph{meaning} \\
\hline\hline
0      & dword    & sh\_size    & sections header size in bytes including this field (32 in this version) \\
4      & dword    & t\_start    & start in file of text section \\
8      & dword    & t\_size     & size in bytes of text section \\
12     & dword    & d\_start    & start in file of data section \\
16     & dword    & d\_size     & size in bytes of data section \\
20     & dword    & b\_size     & size in bytes of bss section \\
24     & dword    & m\_start    & start in file of method table section \\
28     & dword    & m\_size     & size in bytes of method table section \\
\hline
\end{tabular}
\end{center}


\subsection{Text Info}
\label{sec:textinfo}

\par \emph{Text Info - optional}

\par Free-form file string info in form KEY=value, e.g.
\par NAME=my\_funky\_comp\emph{binary 0}COPYRIGHT=2002 (c) johndoe\emph{binary 0}

\par Standard defines some of KEY names for easy human/machine identification.
Remember that key names are case sensitive! AUTHOR and Author are different!

\begin{center}
\par Key Names summary
\begin{tabular}{|l|l|}\hline
\emph{name} & \emph{meaning} \\
\hline\hline
NAME           & comp name (e.g. "my\_funky\_comp") \\
FILENAME       & original comp file name (e.g. "my\_funky.comp") \\
DESCRIPTION    & short textual description of what this comp is/does \\
COPYRIGHT      & short copyright information (don't put COPYING file in here... \\
               & "Copyright (c) YEAR, AUTHOR" will be more than enough) \\
LICENSE        & license (something like "Distributed under BSD License" or "GPL") \\
AUTHOR         & comp author name and email (e.g. "John Doe aka johndoe <jdoe@funny.info>") \\
RELEASE        & file release version (e.g. "1.0.2.4") \\
\hline
\end{tabular}
\end{center}


\subsection{Overlay Data Sections Header}
\label{sec:odshdr}

\begin{center}
\par Overlay Data Sections Header - optional
\begin{tabular}{|l|l|l|l|}\hline
\emph{off} & \emph{size} & \emph{name} & \emph{meaning} \\
\hline\hline
0      & dword    & count      & number of overlay data sections in the header \\
4      & dword    & start      & start address of ODS in memory \\
8      & dword    & size       & size of ODS in memory \\
       &          &            & the above two fields are repeated count times. \\
\hline
\end{tabular}
\end{center}

\par Overlay data sections have no associated data. Instead, they are allocated "on-top" of the
pre-existing memory arena, for example DMA buffer or a video framebuffer.


\small

%\bibliographystyle{plain}
%\bibliography{odin}

\end{document}
