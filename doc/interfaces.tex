
\documentclass[onecolumn]{article}
\usepackage{graphicx}

\begin{document}
\title{\Large\bf Odin OS System-wide interfaces}
\author{Stanislav Karchebny, \\
        email: \texttt{berk@madfire.net}}
\maketitle

\section{Introduction}
\label{sec-intro}

\par This document describes component interfaces used by most of Odin applications, starting from
generic interfaces such as Runnable (implemented by all programs) and down to more task-specific like
Registry interface for supporting various configuration files.

\section{Differences between OMG IDL and Odin IDL}

\subsection{Character set}

\par While OMG IDL uses ISO Latin-1 (8859.1) character set with 114 alphabetic characters distinguished,
Odin IDL is completely i18n-friendly and supports UTF-8 encoded characters.

\par While this is possible to define identifiers in your native language, there are some rules that
will help make programmers' life easier.

FIXME: Character Literals values with respect to UTF-8 encoding (?)

\par Parameter types include int64 and uint64. C-compatible but long-to-spell 'unsigned int' etc. are
replaced with shorter int16, uint16, int32, uint32, int64 and uint64 types.

\par When method is passed unbounded string or unbounded sequence as inout parameter, unlike IDL the
out-going parameter may exceed in-coming parameter in length. (OMG IDL 3.10.2)

\par 'void' alone in method parameters may designate empty parameters list. (OMG IDL 3.14)

\par '/' is used instead of '.' to partition the namespace. (OMG IDL 3.10.4)

\par Variable number of method parameters is allowed using ellipsis ('...') operator.
(!!) Or, should we better support sequence<> _instead_?

\section{Basic interfaces}

\section{Common interfaces}

\section{Specific interfaces}

Interface Definition Language: Static access to interfaces
Interface Repository: Run-time access to interfaces

\small

\bibliographystyle{plain}
\bibliography{odin}

\end{document}
